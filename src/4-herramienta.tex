\chapter{Herramienta}

\section{Objetivos}

- objetivos/requerimientos
    - descripción de qué queríamos que hiciera la herramienta
        - acceso a servidor de cómputo especializado, a elasticsearch, etc.
        - consultas en el corpus
        - entrenamiento y evaluación de vectores
        - descarga, etc.
    - funcionalidades que debía tener, etc.
    - por qué hacerla
    - parecido a sección equivalente del corpus


\section{Diseño de la Solución}

- diseño llegado
    - descripción de funcionalidades de la solución
    - general primero (o no)
    - ir pantalla por pantalla describiendo las principales (o todas) funcionalidades


\section{Implementación}

\subsection{Arquitectura General}

- arq general de la solución (overview): diagrama ES + PSQL + BACKEND + FRONTEND + JOBQUEUE


\subsection{Backend}

- overview: de qué está compuesto (entrenamiento, api, job queue, etc.)

- data model: manejo de resultados, de embeddings (tipo, parámetros, query, archivos en FS, etc.),
              de archivos, de jobs (``como se detallará en la siguiente sección'')

- manejo de jobs y job queue con celery, redis; progress report; por qué jobs (demoran mucho en entrenar), etc.

- entrenamiento: librerías usadas para cada algoritmo
    - word2vec-related (SG y CBOW): gensim
    - glove: implementación de los autores, comunicación con subprocess
    - ppmi+svd: implementación propia

- evaluación:
    - cómo se hizo
    - tipos de tests que existen, cómo se implementa cada uno y qué formato de archivo esperan

- API e interacción con el resto (acceso a modelos, corpus search, entrenamiento, etc.)
    - objetivo:
        - brindar interfaz http restful que permita interactuar con el backend
        - que intente lo más posible esconder las complejidades del backend
          (la persona que usa la api no tiene que saber de celery, redis, postgres, elasticsearch,
           las implementaciones de los algoritmos que existen, ni de evaluaciones; sólo se maneja
           a nivel de los modelos)
        - a ser consumida por el frontend
    - ir sección a sección:
        - corpus search
        - consulta de modelos básicos
        - job management
        - deletion, etc.

---


\subsection{Frontend}

- frontend