\begin{abstract}
En el presente documento se describe el trabajo realizado en el marco del proyecto de grado
``Representación de Palabras en Espacios de Vectores''. Dicho proyecto buscó realizar una evaluación y
comparación de las principales técnicas para la generación de representaciones vectoriales de
palabras aplicadas al idioma español, haciendo especial énfasis en la experimentación directa con
las mismas.

Dado que las técnicas estudiadas requiren grandes cantidades de texto para su buen funcionamiento,
se investigaron y aplicaron técnicas para la extracción masiva de texto de Internet. A su vez, se
construyó una herramienta web para simplificar las tareas de entrenamiento y evaluación de los
vectores generados.  Dicha herramienta permite también realizar consultas complejas sobre el corpus
construido y centraliza todas las tareas de extracción de texto de Internet.

Para la evaluación de las representaciones vectoriales se tradujeron al español los principales
conjuntos de prueba de la literatura y se propusieron además una serie de casos de prueba novedosos
como complemento a los ya existentes.

Como resultado se presenta un corpus en idioma español de más de seis mil millones de palabras,
junto con una herramienta de código abierto que permite generar y evaluar vectores de palabras
utilizando diversos algoritmos del estado del arte del campo. La evaluación de las representaciones
generadas arrojó, asimismo, resultados comparables a los obtenidos en la literatura para el idioma
inglés.
\end{abstract}